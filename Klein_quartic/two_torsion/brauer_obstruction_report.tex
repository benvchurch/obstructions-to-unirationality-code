\documentclass[11pt]{amsart}

\usepackage{amsmath, amssymb, amsthm}
\usepackage{mathrsfs}
\usepackage{hyperref}
\usepackage{enumitem}

\theoremstyle{plain}
\newtheorem{theorem}{Theorem}[section]
\newtheorem{proposition}[theorem]{Proposition}
\newtheorem{lemma}[theorem]{Lemma}
\newtheorem{corollary}[theorem]{Corollary}

\theoremstyle{definition}
\newtheorem{definition}[theorem]{Definition}
\newtheorem{example}[theorem]{Example}

\theoremstyle{remark}
\newtheorem{remark}[theorem]{Remark}

\DeclareMathOperator{\Pic}{Pic}
\DeclareMathOperator{\Br}{Br}
\DeclareMathOperator{\Gal}{Gal}
\DeclareMathOperator{\Aut}{Aut}
\DeclareMathOperator{\Div}{div}
\DeclareMathOperator{\Nm}{N}
\DeclareMathOperator{\Sel}{Sel}
\DeclareMathOperator{\Sha}{\cyrsha}
\DeclareMathOperator{\rk}{rk}
\DeclareMathOperator{\ord}{ord}
\DeclareMathOperator{\Spec}{Spec}
\DeclareMathOperator{\im}{im}
\DeclareMathOperator{\coker}{coker}

\newcommand{\QQ}{\mathbb{Q}}
\newcommand{\ZZ}{\mathbb{Z}}
\newcommand{\CC}{\mathbb{C}}
\newcommand{\RR}{\mathbb{R}}
\newcommand{\FF}{\mathbb{F}}
\newcommand{\PP}{\mathbb{P}}
\newcommand{\Aff}{\mathbb{A}}
\newcommand{\Gm}{\mathbb{G}_m}
\newcommand{\Qbar}{\overline{\QQ}}
\newcommand{\Cbar}{\overline{C}}

\title{Nontrivial Brauer obstructions to descent of \'etale double covers}

\author{}
\date{\today}

\begin{document}

\begin{abstract}
We study the Brauer obstruction for the Fermat quartic
$C\colon x^4 + y^4 + z^4 = 0$ over~$\QQ$.
A quadric decomposition over $K = \QQ(\sqrt{-3})$ produces a $2$-torsion
class $\eta \in J[2](\QQ) \setminus V_{\mathrm{rat}}$, corresponding to an
\'etale double cover $D \to C$ of genus~$5$.
By computing the descent cocycle $\lambda = f \cdot \sigma(f) = -2/3$,
which is negative and hence not a norm from~$K^*$, we establish
$\delta(\eta) \neq 0$ in $\Br(\QQ)[2]$: the cover $D \to C$ does not
descend to~$\QQ$.
However, the abstract curve $D$ \emph{does} admit a $\QQ$-model.
The twist $C_2\colon x^4 + y^4 - z^4 = 0$, isomorphic to $C$ over
$\QQ(\zeta_8)$, has a rational point $(1:0:1)$, and the transported
class $\varphi(\eta)$ remains in $J[2](C_2)(\QQ)$.
Since a rational point rigidifies the Picard scheme, the Brauer
obstruction vanishes on~$C_2$, so the corresponding cover $D' \to C_2$
descends to~$\QQ$, giving a $\QQ$-model of~$D$.

We also exhibit a ``generic'' smooth plane quartic $C'$ with
$\Aut(C'_{\overline{\QQ}}) = 1$ possessing phantom $2$-torsion:
$J[2](\QQ) \cong \ZZ/2\ZZ$ is generated by a class $\eta'$ with
Brauer obstruction $\delta(\eta') = (-1, -3)_\QQ$, ramified at $\infty$
and~$3$. This shows that phantom $2$-torsion is not specific to curves
with large automorphism groups.
All computations were performed in Magma~\cite{Magma}.
\end{abstract}

\maketitle

%% ========================================================================
\section{Introduction}
%% ========================================================================

Let $C \subset \PP^2_\QQ$ be the Fermat quartic curve defined by
\[
  C\colon\; x^4 + y^4 + z^4 = 0.
\]
This is a smooth curve of genus $3$ with Jacobian $J$.
A quadric decomposition of the defining equation over the quadratic field
$K = \QQ(\sqrt{-3})$ produces a $2$-torsion class
$\eta \in J[2](\QQ) \setminus V_{\mathrm{rat}}$,
where $V_{\mathrm{rat}} \subset J[2](\QQ)$ is the subgroup arising from
rational bitangent lines. The class $\eta$ corresponds to an \'etale double
cover $D \to C$ of genus~$5$, defined over $K$.
The cover $D \to C$ does not descend to~$\QQ$: the Brauer obstruction
$\delta(\eta) \neq 0$ prevents it. However, the abstract curve $D$
\emph{does} admit a $\QQ$-model, obtained by transporting $\eta$ to a
twist of $C$ that has a rational point.

More precisely, the Hochschild--Serre spectral sequence provides a
connecting homomorphism
\[
  \delta\colon \Pic(\Cbar)^{G_\QQ} \longrightarrow \Br(\QQ),
\]
whose kernel is $\Pic(C)$, the group of line bundles actually defined
over $\QQ$. A class $\eta \in J[2](\QQ) \subset \Pic^0(\Cbar)^{G_\QQ}$
with $\delta(\eta) \neq 0$ witnesses a Galois-invariant line bundle that
does not descend to~$\QQ$.

\begin{theorem}\label{thm:main}
Let $C\colon x^4 + y^4 + z^4 = 0$ and let $J = \operatorname{Jac}(C)$.
Then $J[2](\QQ) \cong (\ZZ/2\ZZ)^3$, and the obstruction map
\[
  \delta\colon J[2](\QQ) \longrightarrow \Br(\QQ)[2]
\]
has kernel $V_{\mathrm{rat}} \cong (\ZZ/2\ZZ)^2$, the subgroup spanned by
differences of rational bitangent contact divisors.
In particular, $\delta$ is nonzero: there exists a Galois-invariant
$2$-torsion line bundle on $C_{\Qbar}$ that does not descend to $\QQ$.
\end{theorem}

The proof uses the quadric decomposition method of
Bruin~\cite{Bruin2008} over $K = \QQ(\sqrt{-3})$, followed by an
explicit descent cocycle computation.

%% ========================================================================
\section{Background}
%% ========================================================================

%% ------------------------------------------------------------------------
\subsection{The Brauer group and the Hochschild--Serre spectral sequence}
%% ------------------------------------------------------------------------

Let $X$ be a smooth projective variety over a field $k$ with separable
closure $\bar{k}$ and absolute Galois group $G_k = \Gal(\bar{k}/k)$.
The \emph{Brauer group} $\Br(X) := H^2_{\text{\'et}}(X, \Gm)$
fits into a filtration
\[
  \Br_0(X) \subset \Br_1(X) \subset \Br(X),
\]
where $\Br_0(X) := \im(\Br(k) \to \Br(X))$ and
$\Br_1(X) := \ker(\Br(X) \to \Br(X_{\bar{k}}))$
is the \emph{algebraic Brauer group}~\cite{Grothendieck1968}.

The Hochschild--Serre spectral sequence
\[
  E_2^{p,q} = H^p(G_k,\, H^q(X_{\bar{k}},\, \Gm))
    \;\Longrightarrow\; H^{p+q}(X,\, \Gm)
\]
yields, via the identification $H^1(X_{\bar{k}}, \Gm) = \Pic(X_{\bar{k}})$
and Hilbert's Theorem~90 ($H^1(G_k, \bar{k}^*) = 0$), the exact
sequence~\cite[Theorem~5.5.1]{Poonen2017}
\begin{equation}\label{eq:HS}
  0 \to \Pic(X) \to \Pic(X_{\bar{k}})^{G_k}
    \xrightarrow{\;\delta\;} \Br(k)
    \to \Br_1(X) \to H^1(G_k,\, \Pic(X_{\bar{k}}))
    \to H^3(G_k,\, \bar{k}^*).
\end{equation}
For a smooth projective curve $C/k$, the group $\Br(C_{\bar{k}})$
vanishes~\cite[Corollary~6.4.6]{Poonen2017}, so
$\Br_1(C) = \Br(C)$.

The connecting homomorphism $\delta$ in~\eqref{eq:HS} sends a
Galois-invariant line bundle class $[\mathscr{L}] \in \Pic(X_{\bar{k}})^{G_k}$
to the Brauer class measuring the obstruction to descending $\mathscr{L}$
from $\bar{k}$ to $k$. Its kernel is precisely $\Pic(X)$, the subgroup
of classes representable by line bundles defined over $k$.

%% ------------------------------------------------------------------------
\subsection{Descent of line bundles over quadratic extensions}
%% ------------------------------------------------------------------------

For a quadratic extension $K/k$ with $\Gal(K/k) = \{1, \sigma\}$, the
obstruction to descending a $K$-defined line bundle $\mathscr{L}$ to $k$ is
computed as follows~\cite[\S5.4]{Skorobogatov2001}.
Suppose $\mathscr{L}$ is Galois-invariant, i.e.,
$\sigma^*\mathscr{L} \cong \mathscr{L}$.
Choose an isomorphism $\psi\colon \sigma^*\mathscr{L} \xrightarrow{\sim} \mathscr{L}$.
The \emph{descent cocycle} is
\[
  \lambda \;:=\; \psi \circ \sigma^*(\psi)
    \;\in\; \Aut(\mathscr{L}) \;=\; K^*.
\]
One checks that $\sigma(\lambda) = \lambda$, so $\lambda \in k^*$.
Replacing $\psi$ by $c \cdot \psi$ for $c \in K^*$ changes $\lambda$ to
$\Nm_{K/k}(c) \cdot \lambda$. Hence the obstruction class
\[
  [\lambda] \;\in\; k^*/\Nm_{K/k}(K^*) \;\cong\; \Br(K/k)
    \;\hookrightarrow\; \Br(k)[2]
\]
is well-defined. The line bundle $\mathscr{L}$ descends to $k$ if and
only if $\lambda \in \Nm_{K/k}(K^*)$.

\begin{remark}
For $K = \QQ(\sqrt{-3})$, the norm form is
$\Nm(a + b\sqrt{-3}) = a^2 + 3b^2$, which is non-negative for all
$a, b \in \QQ$. Therefore, $\lambda \in \QQ^*$ is a norm from $K^*$ only
if $\lambda > 0$.
\end{remark}

%% ------------------------------------------------------------------------
\subsection{Quadric decompositions and 2-torsion on Jacobians}
\label{sec:quadric}
%% ------------------------------------------------------------------------

Let $C \subset \PP^2$ be a smooth plane quartic defined by a degree-$4$
form $F(x,y,z)$. A \emph{quadric decomposition} of $F$ over a field
$L \supset k$ is an identity
\begin{equation}\label{eq:quadric}
  F = Q_1 Q_3 - Q_2^2,
\end{equation}
where $Q_1, Q_2, Q_3 \in L[x,y,z]$ are homogeneous of degree $2$.
Such a decomposition determines a $2$-torsion divisor class on $J = \operatorname{Jac}(C)$
as follows~\cite{Bruin2008}.

Restricting $Q_1$ to $C$ gives a rational function
$q_1 = Q_1|_C \in L(C)^*$. The identity~\eqref{eq:quadric} implies
$q_1 q_3 = q_2^2$, so $\Div(q_1) + \Div(q_3) = 2\,\Div(q_2)$. In
particular, $\Div(q_1)$ has all-even multiplicities (since $\Div(q_1 q_3)$
does), and the class
\[
  \eta \;:=\; \bigl[\tfrac{1}{2}\Div(q_1)\bigr]
    \;\in\; \Pic^0(C_{\bar{k}})
\]
satisfies $2\eta = [\Div(q_1)] = 0$ in $\Pic^0$
(as $q_1$ is a rational function). Thus $\eta \in J[2]$.

\begin{remark}
The class $\eta$ is the correct formula for the $2$-torsion element:
one halves \emph{all} multiplicities (both zeros and poles) of $\Div(q_1)$.
An alternative formula sometimes seen in the literature,
$[\tfrac{1}{2}\Div_+(q_1) - \tfrac{1}{2}\Div_+(q_3)]$
(halving only the positive parts), equals $[\Div(q_2/q_3)]$, which is
always principal and hence trivial.
\end{remark}

%% ========================================================================
\section{The Fermat quartic: basic properties}
%% ========================================================================

Let $C\colon x^4 + y^4 + z^4 = 0$ over $\QQ$.

%% ------------------------------------------------------------------------
\subsection{The Jacobian and its 2-torsion}
%% ------------------------------------------------------------------------

The curve $C$ has genus $g = 3$. Its Jacobian $J$ is isogenous (over $\Qbar$)
to $E^3$, where $E\colon y^2 = x^3 - x$ is the elliptic curve with
CM by $\ZZ[i]$ and $j$-invariant $1728$~\cite{Koblitz1993}.
The full $2$-torsion group is $J[2](\Qbar) \cong (\ZZ/2\ZZ)^6$,
with $2$-torsion field $\QQ(\zeta_8)$~\cite{Zarhin2000}.
Over $\QQ$, the Galois-invariant subgroup is
\[
  J[2](\QQ) \;\cong\; (\ZZ/2\ZZ)^3.
\]

%% ------------------------------------------------------------------------
\subsection{Bitangent lines and $V_{\mathrm{rat}}$}
%% ------------------------------------------------------------------------

A smooth plane quartic of genus $3$ has exactly $28$ bitangent lines
over $\bar{k}$, and their pairwise contact divisor differences generate
$J[2](\bar{k})$. The Fermat quartic has exactly four rational
bitangent lines:
\[
  x + y + z = 0, \quad x + y - z = 0, \quad x - y + z = 0, \quad x - y - z = 0.
\]
The pairwise differences of the half-contact-divisors span a subgroup
\[
  V_{\mathrm{rat}} \;\cong\; (\ZZ/2\ZZ)^2 \;\subset\; J[2](\QQ).
\]
These classes lie in $\ker(\delta)$, since the corresponding line bundles
are visibly defined over $\QQ$ (they arise from intersecting $C$ with
rational lines, giving effective divisors in $\Div(C)$).

Since $\dim_{\FF_2} J[2](\QQ) = 3$ and
$\dim_{\FF_2} V_{\mathrm{rat}} = 2$, there is a ``missing direction''
$\eta_0 \in J[2](\QQ) \setminus V_{\mathrm{rat}}$, and the content of
Theorem~\ref{thm:main} is that $\delta(\eta_0) \neq 0$.

%% ========================================================================
\section{The quadric decomposition over $\QQ(\sqrt{-3})$}
%% ========================================================================

%% ------------------------------------------------------------------------
\subsection{Nonexistence over $\QQ$}
%% ------------------------------------------------------------------------

A computational search over $\QQ$ (testing all quadratic forms $Q_2$
with integer coefficients in $[-5, 5]$, a total of $885{,}780$ candidates)
finds \emph{no} decomposition $F = Q_1 Q_3 - Q_2^2$ over~$\QQ$
that produces a class outside $V_{\mathrm{rat}}$.

%% ------------------------------------------------------------------------
\subsection{Rational decompositions and $V_{\mathrm{rat}}$}
\label{sec:rational_decomps}
%% ------------------------------------------------------------------------

While no rational quadric decomposition produces the class~$\eta$,
rational decompositions \emph{do} exist with a modified scaling.
The identity
\begin{equation}\label{eq:rational_decomp}
  2(x^4 + y^4 + z^4)
    \;=\; \bigl((a-b)^2 + c^2\bigr)\bigl((a+b)^2 + c^2\bigr)
    \;+\; (a^2 + b^2 - c^2)^2
\end{equation}
holds for any permutation $\{a,b,c\}$ of $\{x,y,z\}$, giving three
decompositions of the form $2F = Q_1 Q_3 + Q_2^2$ over~$\QQ$.
The three choices produce exactly the nonzero elements
of~$V_{\mathrm{rat}}$:
\[
  \begin{array}{lcl}
    Q_1 = (x-y)^2 + z^2 & \longrightarrow & v_2, \\
    Q_1 = (y-z)^2 + x^2 & \longrightarrow & v_1 + v_2, \\
    Q_1 = (x-z)^2 + y^2 & \longrightarrow & v_1.
  \end{array}
\]
On $C$, the relation $2F = Q_1 Q_3 + Q_2^2$ becomes
$Q_1 Q_3 = -Q_2^2$, so $\Div(q_1)$ has all-even multiplicities and the
half-divisor class $[\frac{1}{2}\Div(q_1)]$ is well-defined regardless
of the scaling factor.

These decompositions account for all of
$\ker(\delta) = V_{\mathrm{rat}}$ via quadric methods: the classes
$v_1$, $v_2$, and $v_1 + v_2$ are realized by $\QQ$-rational quadrics,
while $\eta$ and its $V_{\mathrm{rat}}$-translates require an
extension of~$\QQ$.

%% ------------------------------------------------------------------------
\subsection{Decomposition over $K = \QQ(\sqrt{-3})$}
%% ------------------------------------------------------------------------

Let $K = \QQ(\sqrt{-3})$ with $w = \sqrt{-3}$. The identity
\begin{equation}\label{eq:decomp}
  x^4 + y^4 + z^4
    \;=\; \bigl(2x^2 + (1 - w)y^2 + (1 + w)z^2\bigr)
          \bigl(x^2 + \tfrac{1+w}{2}\, y^2 + \tfrac{w-1}{2}\, z^2\bigr)
    \;-\; \bigl(x^2 + y^2 + w\, z^2\bigr)^2
\end{equation}
gives a quadric decomposition~\eqref{eq:quadric} over $K$ with
\[
  Q_1 = 2x^2 + (1-w)y^2 + (1+w)z^2, \quad
  Q_2 = x^2 + y^2 + w\, z^2.
\]

%% ------------------------------------------------------------------------
\subsection{Alternative decomposition over $\QQ(i)$}
\label{sec:Qi_decomp}
%% ------------------------------------------------------------------------

Since the Brauer class $\delta(\eta) = (-\frac{2}{3}, -3)_\QQ$ has local
invariants $\tfrac{1}{2}$ at $v = \infty$ and $v = 2$ only
(see~\S\ref{sec:tate}), any quadratic extension that splits both places
must kill the obstruction---and hence must support a quadric decomposition
producing the class~$\eta$. The field $\QQ(i)$ has this property:
$\QQ(i)$ is complex (splitting $\infty$) and $2$ ramifies in $\QQ(i)$
(splitting the local Brauer class at $2$, since any quadratic extension
of $\QQ_2$ splits the unique nontrivial element of $\Br(\QQ_2)[2]$).

A computational search confirms the existence of a decomposition over
$\QQ(i)$. The simplest example is
\begin{equation}\label{eq:decomp_Qi}
  x^4 + y^4 + z^4
    \;=\; (2x^2 + 2iz^2)(x^2 + iy^2)
    \;-\; (x^2 + iy^2 + iz^2)^2,
\end{equation}
with $Q_1 = 2x^2 + 2iz^2$, $Q_2 = x^2 + iy^2 + iz^2$,
$Q_3 = x^2 + iy^2$, where $i = \sqrt{-1}$.

A computation over $\FF_{13}$ and $\FF_{37}$ (primes $\equiv 1 \pmod{12}$
where both $\sqrt{-1}$ and $\sqrt{-3}$ exist) verifies that the
half-divisor class $[\frac{1}{2}\Div(q_1)]$ from~\eqref{eq:decomp_Qi}
is \textbf{equal} to the class $\eta$ produced by the $\QQ(\sqrt{-3})$
decomposition~\eqref{eq:decomp}, and is not in $V_{\mathrm{rat}}$.

\begin{remark}\label{rem:splitting_fields}
The decomposition~\eqref{eq:decomp_Qi} is in some ways more natural
than~\eqref{eq:decomp}: the factors $Q_1 = 2(x^2 + iz^2)$ and
$Q_3 = x^2 + iy^2$ visibly exploit the Gaussian factorization of
sums of squares. More generally, a quadric decomposition producing
$\eta$ exists over any quadratic extension $\QQ(\sqrt{d})$ that
splits both ramified places of $\delta(\eta)$. Since $\delta(\eta)$
is ramified at $\infty$ and $2$, the extension must be
\emph{imaginary} (to split $\infty$) and $2$ must \emph{not split}
in $\QQ(\sqrt{d})$ (equivalently $d \not\equiv 1 \pmod{8}$, so that
the local degree $[\QQ_2(\sqrt{d}) : \QQ_2] = 2$ kills the $2$-local
Brauer class). The imaginary quadratic fields $\QQ(i)$,
$\QQ(\sqrt{-2})$, and $\QQ(\sqrt{-3})$ all satisfy these conditions.
On the other hand, $\QQ(\sqrt{-7})$ does \emph{not}: since
$-7 \equiv 1 \pmod{8}$, the prime $2$ splits in $\QQ(\sqrt{-7})$,
leaving the local Brauer class at~$2$ intact.
\end{remark}

%% ------------------------------------------------------------------------
\subsection{Identification of the 2-torsion class}
%% ------------------------------------------------------------------------

To identify the class $\eta = [\frac{1}{2}\Div(q_1)] \in J[2]$,
we reduce modulo $3$. Since $w = \sqrt{-3} \equiv 0 \pmod{3}$,
the decomposition~\eqref{eq:decomp} reduces over $\FF_3$ (after a
coordinate permutation $(x,y,z) \mapsto (y,z,x)$) to the decomposition
with $Q_2 = y^2 + z^2$.

An exhaustive computation of all quadric decompositions over $\FF_3$
yields four distinct $J[2]$ classes. Writing $J[2](\FF_3) = \langle
e_1, e_2, e_3 \rangle \cong (\ZZ/2\ZZ)^3$, three of these classes
($e_1$, $e_2$, and $e_1 + e_2$) lie in
$V_{\mathrm{rat}} = \langle e_1, e_2 \rangle$, and the fourth is
\[
  \eta \;=\; e_1 + e_2 + e_3 \;\notin\; V_{\mathrm{rat}}.
\]
This is the ``missing'' class.

%% ------------------------------------------------------------------------
\subsection{Bitangent decompositions over $\QQ(\sqrt{-2})$}
\label{sec:bitangent_decomps}
%% ------------------------------------------------------------------------

The four rational bitangent lines $L_i$ pair into six products
$L_i L_j$, each a reducible conic. Over $\QQ(\sqrt{-2})$, each product
gives a decomposition $F = L_i L_j \cdot Q_3 - Q_2^2$ with
$Q_2 = \sqrt{-2}\, P$ for a rational quadric~$P$. Explicitly, with
$L_1 = x+y+z$, $L_2 = x+y-z$, $L_3 = x-y+z$, $L_4 = x-y-z$:
\begin{align*}
  F &= L_1 L_2 \cdot \bigl(-(x{+}y)^2 - z^2\bigr)
    - \bigl(\sqrt{-2}\,(x^2{+}xy{+}y^2)\bigr)^2, \\
  F &= L_3 L_4 \cdot \bigl(-(x{-}y)^2 - z^2\bigr)
    - \bigl(\sqrt{-2}\,(x^2{-}xy{+}y^2)\bigr)^2,
\end{align*}
and similarly for the other four products (obtained by permuting the
roles of $x, y, z$).

The six bitangent products fall into three $S_3$-orbits, each
corresponding to a nonzero element of~$V_{\mathrm{rat}}$:
\[
  \{L_1 L_2,\, L_3 L_4\} \to v_2, \qquad
  \{L_1 L_3,\, L_2 L_4\} \to v_1, \qquad
  \{L_1 L_4,\, L_2 L_3\} \to v_1 + v_2.
\]
Each pair gives the same $2$-torsion class as the corresponding
rational decomposition from~\eqref{eq:rational_decomp}.

%% ------------------------------------------------------------------------
\subsection{The $\mathrm{GL}_2$ orbit structure}
\label{sec:gl2_orbits}
%% ------------------------------------------------------------------------

The equation $F = Q_1 Q_3 - Q_2^2$ can be written as $\det M = F$
where $M = \bigl(\begin{smallmatrix} Q_1 & Q_2 \\ Q_2 & Q_3
\end{smallmatrix}\bigr)$. The group $\mathrm{GL}_2$ acts by
$M \mapsto g M g^{\!\top}$, preserving $\det M$ up to $(\det g)^2$;
elements with $(\det g)^2 = 1$ preserve the decomposition.

Over $\QQ(i)$, all $12$ distinct decompositions with $\lambda = 1$
(three $S_3$-basic families and their $\mathrm{GL}_2$ transforms) lie
in a \textbf{single} $\mathrm{SL}_2(\QQ(i))$-orbit. This is the
unique orbit producing the class~$\eta$; the classes in
$V_{\mathrm{rat}}$ require $\sqrt{2} \notin \QQ(i)$ and hence are not
accessible at this scaling.

The rational decompositions (\S\ref{sec:rational_decomps}) and
bitangent decompositions (\S\ref{sec:bitangent_decomps}) both produce
$V_{\mathrm{rat}}$ classes but with different scalings. In the common
normalization $-2F = Q_1 Q_3 - Q_2^2$ (absorbing the $\sqrt{-2}$ from
the bitangent decomposition), both become rational, and the two
decompositions for each theta characteristic are
$\mathrm{GL}_2(\QQ(\sqrt{-2}))$-conjugate but \emph{not}
$\mathrm{GL}_2(\QQ)$-conjugate. For instance, the decomposition with
$Q_1 = (x-y)^2 + z^2$ maps to the one with
$Q_1 = (x+y)^2 - z^2 = L_1 L_2$ via
\[
  g = \begin{pmatrix}
    \frac{\sqrt{-2}}{2} & -\frac{\sqrt{-2}}{2} \\[4pt]
    0 & \sqrt{-2}
  \end{pmatrix},
  \qquad \det g = -1.
\]
The $\sqrt{-2}$ obstruction to $\QQ$-conjugacy reflects the field of
definition of the bitangent decomposition itself.

%% ========================================================================
\section{The descent cocycle}
%% ========================================================================

%% ------------------------------------------------------------------------
\subsection{Galois invariance of $\eta$}
%% ------------------------------------------------------------------------

Since $\eta$ arises from a decomposition over $K = \QQ(\sqrt{-3})$, it
is \emph{a priori} an element of $J[2](K)$. To apply descent, we first
verify that $\sigma(\eta) = \eta$, where $\sigma$ is the nontrivial
element of $\Gal(K/\QQ)$ acting by $w \mapsto -w$.

The conjugate decomposition has
$\sigma(Q_1) = 2x^2 + (1+w)y^2 + (1-w)z^2$.
A direct computation in the class group of the function field of $C$
over $\FF_7$ (where $\sqrt{-3} \equiv 2$ and
$\sigma(\sqrt{-3}) \equiv 5$) confirms
$[\tfrac{1}{2}\Div(q_1)] = [\tfrac{1}{2}\Div(\sigma(q_1))]$ in $J[2](\FF_7)$.

Since the reduction map $J[2](\QQ) \hookrightarrow J[2](\FF_7)$ is
injective (as $7$ is a prime of good reduction), this implies
$\sigma(\eta) = \eta$ globally. Thus
$\eta \in J[2](\QQ) \setminus V_{\mathrm{rat}}$.

%% ------------------------------------------------------------------------
\subsection{Setup of the cocycle computation}
%% ------------------------------------------------------------------------

Working in the function field $K(C)$ with affine coordinates $t = x/z$,
$u = y/z$ satisfying $u^4 + t^4 + 1 = 0$, we set
\begin{align*}
  q_1 &= 2t^2 + (1-w)u^2 + (1+w), \\
  \sigma(q_1) &= 2t^2 + (1+w)u^2 + (1-w).
\end{align*}
A direct expansion using $w^2 = -3$ yields the \emph{norm identity}
\begin{equation}\label{eq:norm}
  q_1 \cdot \sigma(q_1) \;=\; 4g,
  \qquad g := t^2 u^2 + t^2 - u^2 \;\in\; \QQ(C)^*.
\end{equation}
Geometrically,~\eqref{eq:norm} states that the norm
$\Nm_{K/\QQ}(\eta) = \eta + \sigma(\eta) = [\tfrac{1}{2}\Div(g)]$
is a $\QQ$-rational divisor class, a necessary condition for descent.

The divisors $D := \tfrac{1}{2}\Div(q_1)$ and
$\sigma(D) := \tfrac{1}{2}\Div(\sigma(q_1))$ are well-defined
(all multiplicities of $\Div(q_1)$ and $\Div(\sigma(q_1))$ are even).
Since $\eta = \sigma(\eta)$ in $J[2]$, the divisor $D - \sigma(D)$ is
linearly equivalent to $0$, and there exists
$f \in K(C)^*$ with
\begin{equation}\label{eq:f}
  \Div(f) = D - \sigma(D).
\end{equation}

%% ------------------------------------------------------------------------
\subsection{Computation of $\lambda$}
%% ------------------------------------------------------------------------

Using the Riemann--Roch space $L(\sigma(D) - D)$ over $K(C)$, Magma finds
the unique (up to scalar) function $f$ satisfying~\eqref{eq:f}:
\begin{equation}\label{eq:explicit_f}
  f \;=\; \frac{u^2 + \frac{w}{3}(t^2 + 1)}{t^2 - \frac{w+1}{2}}.
\end{equation}
Applying $\sigma\colon w \mapsto -w$ gives
\[
  \sigma(f) \;=\; \frac{u^2 - \frac{w}{3}(t^2 + 1)}{t^2 + \frac{w - 1}{2}}.
\]

The descent cocycle is $\lambda = f \cdot \sigma(f)$.
Multiplying the numerators:
\begin{align*}
  \Bigl(u^2 + \tfrac{w}{3}(t^2+1)\Bigr)
  \Bigl(u^2 - \tfrac{w}{3}(t^2+1)\Bigr)
  &= u^4 - \tfrac{w^2}{9}(t^2+1)^2 \\
  &= u^4 + \tfrac{1}{3}(t^2+1)^2.
\end{align*}
On $C$, we have $u^4 = -(t^4 + 1)$, so
\[
  u^4 + \tfrac{1}{3}(t^2+1)^2
  = -(t^4+1) + \tfrac{1}{3}(t^4 + 2t^2 + 1)
  = -\tfrac{2}{3}(t^4 - t^2 + 1).
\]
Multiplying the denominators:
\[
  \Bigl(t^2 - \tfrac{w+1}{2}\Bigr)
  \Bigl(t^2 + \tfrac{w-1}{2}\Bigr)
  = t^4 + \tfrac{(w-1)-(w+1)}{2}\,t^2 + \tfrac{1 - w^2}{4}
  = t^4 - t^2 + 1.
\]

Therefore:
\begin{equation}\label{eq:lambda}
  \boxed{\lambda \;=\; f \cdot \sigma(f)
    \;=\; \frac{-\frac{2}{3}(t^4 - t^2 + 1)}{t^4 - t^2 + 1}
    \;=\; -\frac{2}{3}.}
\end{equation}

%% ------------------------------------------------------------------------
\subsection{The norm condition}
%% ------------------------------------------------------------------------

\begin{proposition}\label{prop:not_norm}
The element $\lambda = -2/3$ is not in the image of the norm map
$\Nm_{K/\QQ}\colon K^* \to \QQ^*$ for $K = \QQ(\sqrt{-3})$.
\end{proposition}

\begin{proof}
For $a + b\sqrt{-3} \in K^*$, the norm is
$\Nm(a + b\sqrt{-3}) = a^2 + 3b^2 \geq 0$, with equality only when
$a = b = 0$. Since $-2/3 < 0$, it cannot be a norm.
\end{proof}

By the discussion in \S2.2, this means the line bundle
$\mathscr{L} = \mathscr{O}_C(D)$ on $C_K$ corresponding to $\eta$ does
not descend to $\QQ$, i.e.,
$[\lambda] = [-2/3] \neq 0$ in $\QQ^*/\Nm_{K/\QQ}(K^*)$.

%% ------------------------------------------------------------------------
\subsection{Identification of the Brauer class via Tate cohomology}
\label{sec:tate}
%% ------------------------------------------------------------------------

The cocycle $\lambda$ naturally lives in the Tate cohomology group
$\widehat{H}^0_T(G,\, K^*)$, and we now relate it to the Brauer group
$\widehat{H}^2_T(G,\, K^*) \cong \Br(K/\QQ)$, where
$G = \Gal(K/\QQ) = \langle \sigma \rangle \cong \ZZ/2\ZZ$.

Recall that for a cyclic group $G$ of order $n$ with generator $\sigma$
acting on a $G$-module $M$, the Tate cohomology groups
are~\cite[\S VIII.4]{Serre1979}
\[
  \widehat{H}^0_T(G, M) = M^G / \Nm(M),
  \qquad
  \widehat{H}^{-1}_T(G, M) = \ker(\Nm) / (1 - \sigma) M,
\]
where $\Nm = \sum_{g \in G} g$ is the norm map.
Tate's periodicity theorem~\cite[Theorem~6.2.3]{Neukirch2008} states that
cup product with the canonical generator $u \in \widehat{H}^2_T(G, \ZZ)
\cong \ZZ/n\ZZ$ induces isomorphisms
\begin{equation}\label{eq:tate}
  \widehat{H}^r_T(G, M)
    \;\xrightarrow[\;\sim\;]{\;\cup\, u\;}
  \widehat{H}^{r+2}_T(G, M)
  \qquad \text{for all } r \in \ZZ.
\end{equation}

Applied to $M = K^*$ with $G = \Gal(K/\QQ)$:
\begin{itemize}
  \item $\widehat{H}^0_T(G, K^*) = (K^*)^G / \Nm(K^*)
    = \QQ^* / \Nm_{K/\QQ}(K^*)$, where $\lambda = -2/3$ represents
    a nontrivial class.
  \item $\widehat{H}^2_T(G, K^*) = H^2(G, K^*) = \Br(K/\QQ)$,
    the relative Brauer group.
\end{itemize}
The periodicity isomorphism~\eqref{eq:tate} identifies
$[-2/3] \in \widehat{H}^0_T(G, K^*)$ with a nontrivial element of
$\Br(K/\QQ) \hookrightarrow \Br(\QQ)[2]$.

Explicitly, the isomorphism $\QQ^*/\Nm_{K/\QQ}(K^*) \xrightarrow{\sim}
\Br(K/\QQ)$ sends $[a]$ to the class of the quaternion
algebra $(a, d)_\QQ$ where $K = \QQ(\sqrt{d})$~\cite[\S2.5]{GilleSzamuely2006}.
In our case $d = -3$ and $a = -2/3$, so the Brauer class is
\[
  \delta(\eta) \;=\; (-\tfrac{2}{3},\; -3)_\QQ
    \;\in\; \Br(\QQ)[2].
\]
One computes (via the Hilbert symbol) that this quaternion algebra has
local invariants $\operatorname{inv}_v = 1/2$ at $v = \infty$ and $v = 2$,
and $\operatorname{inv}_v = 0$ at all other places.

%% ------------------------------------------------------------------------
\subsection{Alternative cocycle via $\QQ(i)$}
\label{sec:cocycle_Qi}
%% ------------------------------------------------------------------------

The descent cocycle computation simplifies considerably when performed
over $K' = \QQ(i)$ using the decomposition~\eqref{eq:decomp_Qi}.
Let $\sigma'\colon i \mapsto -i$ denote the nontrivial element of
$\Gal(K'/\QQ)$.

Set $q_1 = 2t^2 + 2i$ and $\sigma'(q_1) = 2t^2 - 2i$ in $K'(C)$.
Their product is
\[
  q_1 \cdot \sigma'(q_1) = (2t^2 + 2i)(2t^2 - 2i) = 4t^4 + 4
    = 4(t^4 + 1) = -4u^4,
\]
using $t^4 + u^4 + 1 = 0$ on~$C$. Hence
$D + \sigma'(D) = 2\,\Div(u)$ (where $D = \tfrac{1}{2}\Div(q_1)$),
and therefore
\begin{equation}\label{eq:f_Qi}
  D - \sigma'(D) \;=\; \Div(q_1) - 2\,\Div(u) \;=\; \Div\!\Bigl(\frac{q_1}{u^2}\Bigr),
\end{equation}
so the function $f = q_1/u^2 = (2t^2 + 2i)/u^2$ satisfies
$\Div(f) = D - \sigma'(D)$. The descent cocycle is
\begin{equation}\label{eq:lambda_Qi}
  \boxed{\lambda' \;=\; f \cdot \sigma'(f)
    \;=\; \frac{q_1 \cdot \sigma'(q_1)}{u^4}
    \;=\; \frac{-4u^4}{u^4}
    \;=\; -4.}
\end{equation}
Since $\Nm_{K'/\QQ}(a + bi) = a^2 + b^2 \geq 0$ and $-4 < 0$,
the cocycle $\lambda'$ is not a norm, confirming $\delta(\eta) \neq 0$
via this second splitting field. The resulting quaternion algebra is
\[
  (-4,\, -1)_\QQ \;=\; (-1,\, -1)_\QQ,
\]
the Hamilton quaternions (since $-4 = -1 \cdot 2^2$ and $\Nm(2) = 4$).
This is the unique quaternion algebra over $\QQ$ ramified at
$\{\infty, 2\}$, consistent with the earlier computation
$(-\frac{2}{3},\, -3)_\QQ$.

\begin{remark}
The $\QQ(i)$ computation avoids the Riemann--Roch step entirely:
the function $f = q_1/u^2$ is obtained by inspection from the
identity $q_1 \cdot \sigma'(q_1) = -4u^4$, and the cocycle
$\lambda' = -4$ follows by a one-line calculation. In contrast,
the $\QQ(\sqrt{-3})$ descent requires finding $f$ via a
Riemann--Roch space computation (equation~\eqref{eq:explicit_f})
and a more involved cancellation to reach $\lambda = -2/3$
(equation~\eqref{eq:lambda}).
\end{remark}

%% ========================================================================
\section{Proof of Theorem~\ref{thm:main}}
%% ========================================================================

\begin{proof}[Proof of Theorem~\ref{thm:main}]
We have shown:
\begin{enumerate}[label=(\roman*)]
  \item The $\QQ$-rational bitangent lines of $C$ span
    $V_{\mathrm{rat}} \cong (\ZZ/2\ZZ)^2 \subset J[2](\QQ)$,
    and $V_{\mathrm{rat}} \subset \ker(\delta)$ since these classes are
    represented by $\QQ$-rational divisors.
  \item The quadric decomposition~\eqref{eq:decomp} over
    $K = \QQ(\sqrt{-3})$ (or equivalently~\eqref{eq:decomp_Qi}
    over $K' = \QQ(i)$) produces a class
    $\eta = e_1 + e_2 + e_3 \in J[2](\QQ) \setminus V_{\mathrm{rat}}$.
  \item The descent cocycle over $K'$ gives
    $\lambda' = -4 \notin \Nm_{K'/\QQ}(K'^*)$ (\S\ref{sec:cocycle_Qi}),
    so the \'etale double cover $D \to C$
    corresponding to $\eta$ does not descend to $\QQ$, and
    $\delta(\eta) \neq 0$ in $\Br(\QQ)[2]$.
\end{enumerate}
Since $J[2](\QQ) = V_{\mathrm{rat}} \oplus \langle \eta \rangle
\cong (\ZZ/2\ZZ)^3$ and $\delta(\eta) \neq 0$, the kernel of $\delta$
restricted to $J[2](\QQ)$ is exactly $V_{\mathrm{rat}}$.
\end{proof}

%% ========================================================================
\section{Why $\delta(\eta)$ obstructs the descent of $D$}
\label{sec:why}
%% ========================================================================

The Brauer class $\delta(\eta) \in \Br(\QQ)[2]$ was defined as the
obstruction to descending a line bundle on $C$. We now explain why
it also obstructs the descent of the \'etale double cover $D$ itself,
giving two independent arguments.

%% ------------------------------------------------------------------------
\subsection{Via the associated line bundle}
%% ------------------------------------------------------------------------

The \'etale double cover $\pi\colon D \to C$ determines a
$2$-torsion line bundle on $C$ as follows. The pushforward
$\pi_* \mathscr{O}_D$ is a rank-$2$ vector bundle on $C$ equipped
with the action of the deck involution $\iota$; it decomposes into
eigensheaves as
\[
  \pi_* \mathscr{O}_D \;=\; \mathscr{O}_C \;\oplus\; \mathscr{L},
\]
where $\mathscr{L}$ is the $(-1)$-eigensheaf, a line bundle satisfying
$\mathscr{L}^{\otimes 2} \cong \mathscr{O}_C$. The isomorphism class
$[\mathscr{L}] \in \Pic(C)[2]$ is exactly the $2$-torsion class
$\eta$.

If $D$ admitted a model over $\QQ$ \emph{as a cover of $C$}, the morphism
$\pi$ and the decomposition of $\pi_*\mathscr{O}_D$ would also be
defined over $\QQ$, and $\mathscr{L}$ would descend to a line bundle in
$\Pic(C)$. But $\delta(\eta) \neq 0$ means precisely that $\mathscr{L}$
does \emph{not} descend. Hence $D$ cannot descend as a cover of $C$.

%% ------------------------------------------------------------------------
\subsection{Via the \'etale fundamental group}
%% ------------------------------------------------------------------------

The cover $D \to C_{\Qbar}$ corresponds to a surjective character
\[
  \varphi\colon \pi_1^{\text{\'et}}(C_{\Qbar}) \twoheadrightarrow \mu_2
\]
with kernel $H = \ker(\varphi) \subset \pi_1^{\text{\'et}}(C_{\Qbar})$.
The Galois invariance $\sigma(\eta) = \eta$ means that $H$ is stable
under the conjugation action of $G_\QQ$ on $\pi_1^{\text{\'et}}(C_{\Qbar})$.

To descend $D$ as a cover of $C$ to $\QQ$, one must extend $H$ to a normal
subgroup of $\pi_1^{\text{\'et}}(C)$ defining a geometrically connected
cover of $C$ over $\QQ$. Equivalently, one must lift $\varphi$ to a character
of $\pi_1^{\text{\'et}}(C)$ itself. The Hochschild--Serre spectral sequence
for \'etale cohomology with $\mu_2$-coefficients gives the exact
sequence~\cite[\S5.3]{Poonen2017}
\begin{equation}\label{eq:mu2_HS}
  H^1(G_\QQ,\, \mu_2) \;\to\; H^1_{\text{\'et}}(C,\, \mu_2)
    \;\to\; H^1_{\text{\'et}}(C_{\Qbar},\, \mu_2)^{G_\QQ}
    \;\xrightarrow{\;d_2\;}\; H^2(G_\QQ,\, \mu_2).
\end{equation}
The class $\varphi \in H^1_{\text{\'et}}(C_{\Qbar}, \mu_2)^{G_\QQ}$ lifts to
$H^1_{\text{\'et}}(C, \mu_2)$ (i.e., $D$ descends as a cover of $C$
to~$\QQ$) if and only if $d_2(\varphi) = 0$.

The Kummer sequence $1 \to \mu_2 \to \Gm \xrightarrow{(\cdot)^2} \Gm \to 1$
on $\Spec(\QQ)$ yields the identification
\begin{equation}\label{eq:kummer_brauer}
  H^2(G_\QQ,\, \mu_2) \;\cong\; \Br(\QQ)[2],
\end{equation}
and the differential $d_2$ in~\eqref{eq:mu2_HS} is identified with the
restriction of the connecting homomorphism $\delta$ from~\eqref{eq:HS}
to $2$-torsion classes. Concretely, under the natural map
$H^1_{\text{\'et}}(C_{\Qbar}, \mu_2) \to \Pic(C_{\Qbar})[2] = J[2](\Qbar)$,
the character $\varphi$ maps to $\eta$, and
\[
  d_2(\varphi) \;=\; \delta(\eta) \;=\; (-\tfrac{2}{3},\, -3)_\QQ
    \;\neq\; 0 \;\in\; \Br(\QQ)[2].
\]
Thus the obstruction to extending $H \subset \pi_1^{\text{\'et}}(C_{\Qbar})$
to a normal subgroup of $\pi_1^{\text{\'et}}(C)$ defining a geometrically
connected cover over $\QQ$ is precisely the Brauer class $\delta(\eta)$.

%% ------------------------------------------------------------------------
\subsection{The automorphism group of $D$}
\label{sec:aut_D}
%% ------------------------------------------------------------------------

We now determine the full automorphism group of $D$ and its field of
definition.

The class $\eta$ is the \emph{unique} $\Aut(C)$-invariant element of
$J[2](\Qbar) \setminus \{0\}$: for every $\sigma \in \Aut(C)$,
the ratio $\sigma^*(q_1)/q_1$ has all-even valuations and principal
half-divisor, so $\sigma^*(\eta) = \eta$. Thus
$\operatorname{Stab}_{\Aut(C)}(\eta) = \Aut(C)$, the full group of
order~$96$, and the exact sequence
\[
  1 \;\to\; \langle \iota \rangle \;\to\; \Aut(D_{\Qbar})
    \;\to\; \Aut(C_{\Qbar}) \;\to\; 1
\]
gives $|\Aut(D_{\Qbar})| = 2 \times 96 = 192$.
The group is identified as
\[
  \Aut(D_{\Qbar}) \;\cong\; C_2^3.S_4
  \quad \bigl(\text{SmallGroup}(192, 181)\bigr),
\]
confirmed by computing $|\Aut(D/\FF_q)| = 192$ at $q = 9, 49, 97, 193, 241$.

To determine the field of definition, we analyse the \emph{lifting
constant}. For each $\sigma \in \Aut(C)$ fixing~$\eta$, write
$\sigma^*(q_1)/q_1 = c_\sigma \cdot h_\sigma^2$ with
$c_\sigma \in k^*/(k^*)^2$; then $\sigma$ lifts to $\Aut(D)$ over~$k$
if and only if $c_\sigma$ is a square. Computations over $\FF_p$ for
$p = 7, 13, 19, 37, 43$ show:
\begin{itemize}
  \item Even permutations ($\operatorname{id}$ and $3$-cycles in the
    $S_3$ factor of $\Aut(C) = (\ZZ/4\ZZ)^2 \rtimes S_3$) have
    $c_\sigma = 1$.
  \item Odd permutations (transpositions) have
    $c_\sigma \equiv -2 \pmod{(\QQ(i)^*)^2}$.
\end{itemize}
The cover $D \to C$ and its deck involution are defined over
$\QQ(i)$, since the Brauer obstruction $\delta(\eta)$ is ramified only
at $\infty$ and~$2$, both of which split in $\QQ(i)$
(see~\S\ref{sec:Qi_decomp}). Over $\QQ(i)$, the even permutations lift
but the transpositions do not (as $-2$ is not a square in $\QQ(i)^*$).
Since $\sqrt{-2} = i\sqrt{2} \in \QQ(i, \sqrt{2}) = \QQ(\zeta_8)$,
we obtain
\[
  |\Aut(D/\QQ(i))| = 96, \qquad
  |\Aut(D/\QQ(\zeta_8))| = 192.
\]

This is confirmed by the LMFDB~\cite{LMFDB}: the family
\texttt{\href{https://www.lmfdb.org/HigherGenus/C/Aut/5.192-181.0.2-3-8}{5.192-181.0.2-3-8}}
lists exactly $4$ refined passports for genus-$5$ curves with
automorphism group $[192, 181]$ and signature $(0;\, 2, 3, 8)$.
These correspond to $4$ geometric points of the scheme $\mathscr{P}$
parametrising marked pairs $(X,\, G \hookrightarrow \Aut(X))$.
The outer automorphism of $G$ pairs the four into two orbits of
size~$2$, giving two isomorphism classes of unmarked curves, so
$\mathscr{P}$ is an \'etale $\QQ$-scheme of degree~$4$.
Since $[\QQ(\zeta_8):\QQ] = 4$ and $\mathscr{P}$ acquires a rational
point over $\QQ(\zeta_8)$---namely the curve $D$ with its full
automorphism group---but not over any proper subfield (the
intermediate fields $\QQ(i)$, $\QQ(\sqrt{2})$, $\QQ(\sqrt{-2})$ each
lack either $\sqrt{-1}$ or $\sqrt{-2}$), we conclude
\[
  \mathscr{P} \;\cong\; \Spec\,\QQ(\zeta_8).
\]

\begin{remark}[Inflation and the descent cocycle]\label{rem:inflation}
The extension
\begin{equation}\label{eq:aut_ext}
  1 \;\to\; \langle \iota \rangle \;\to\; \Aut(D_{\Qbar})
    \;\to\; \Aut(C_{\Qbar}) \;\to\; 1
\end{equation}
is \emph{non-split}: a computation in $\operatorname{SmallGroup}(192,181)$
confirms that no element $g \in \Aut(D_{\Qbar})$ satisfies $g^2 = \iota$,
and the centre $Z(\Aut(D_{\Qbar})) = \langle \iota \rangle$ admits no
complement.

The cover $D \to C$ is defined over $\QQ(i)$
(see~\S\ref{sec:Qi_decomp}), and the descent cocycle from $\QQ(i)$ to
$\QQ$ is $c(\sigma) = \iota$, where
$\sigma\colon i \mapsto -i$ generates $\Gal(\QQ(i)/\QQ)$.
Trivialising this cocycle requires $\alpha \in \Aut(D_{\Qbar})$ with
$\alpha^{-1}\,\sigma(\alpha) = \iota$.
Over $\QQ(i)$ alone, only $96$ of the $192$ geometric automorphisms are
available (the lifts of even permutations), and among these no such
$\alpha$ exists---indeed $\iota$ is central and no element squares
to~$\iota$.

The lifts of odd permutations (transpositions) provide the needed
$\alpha$: their lifting constant $c_\sigma = -2$ requires
$\sqrt{-2} = i\sqrt{2} \in \QQ(\zeta_8) \setminus \QQ(i)$,
and conjugation by $\sigma$ introduces precisely the sign flip
$\sigma(\sqrt{-2}) = -\sqrt{-2}$ that produces
$\alpha^{-1}\,\sigma(\alpha) = \iota$.
This is the mechanism behind the descent of $D$ to $\QQ$ constructed
in~\S\ref{sec:abstract_descent} below: the isomorphism
$\varphi\colon C \xrightarrow{\sim} C_2$ given by
$(x{:}y{:}z) \mapsto (x{:}y{:}\zeta_8\, z)$ involves an odd permutation
of the $4$th-root scalings
(since $\zeta_8^4 = -1$), and its lift to $D$ absorbs the cocycle~$\iota$.
\end{remark}

%% ------------------------------------------------------------------------
\subsection{Descent of $D$ as an abstract curve}
\label{sec:abstract_descent}
%% ------------------------------------------------------------------------

The arguments above show that $D$ does not descend to $\QQ$ \emph{as a
cover of $C$}. We now show that, nevertheless, $D$ \emph{does} descend
to $\QQ$ as an abstract curve.

The key observation is that the deck involution $\iota$ is the unique
involution of $D_{\Qbar}$ whose quotient is a smooth non-hyperelliptic
curve of genus~$3$. It is therefore characterized by an intrinsic
geometric property and must be preserved by any $G_\QQ$-action on
$\Aut(D_{\Qbar})$. Consequently, if $D$ descends to some $D_0/\QQ$,
the involution $\iota$ also descends, and $D_0/\langle \iota \rangle$
is a twist $C'$ of $C$ over $\QQ$, with $D_0 \to C'$ an \'etale double
cover defined over $\QQ$.

Consider the twist $C_2\colon x^4 + y^4 - z^4 = 0$, which is isomorphic
to $C$ via $\varphi\colon (x:y:z) \mapsto (x:y:\zeta_8 z)$ over
$\QQ(\zeta_8)$, where $\zeta_8^4 = -1$. The isomorphism $\varphi$
induces a map $\varphi_*\colon J[2](C) \to J[2](C_2)$, and a
computation over $\FF_{49}$ (where both $\sqrt{-3}$ and $\zeta_8$
exist) verifies that
\[
  \varphi_*(\eta) \;\in\; J[2](C_2)(\QQ).
\]
Specifically, over $\FF_{49}$ the class $\eta$ has coordinates
$(1,0,0,0,1,0)$ in the $J[2]$ basis, and $\varphi_*(\eta) = (0,0,1,0,0,0)$,
which is fixed by $\operatorname{Frob}_7$. Since the $2$-rank of $J(C_2)$
at $p = 7$ equals $3 = \dim J[2](C_2)(\QQ)$, the Frobenius-fixed subspace
equals $J[2](C_2)(\QQ)$, confirming rationality.

\begin{proposition}\label{prop:D_descends}
The abstract curve $D$ is defined over $\QQ$.
\end{proposition}

\begin{proof}
The curve $C_2$ has the rational point $(1:0:1)$, so its Picard scheme
$\Pic_{C_2/\QQ}$ admits a section. This rigidifies the Picard scheme and
implies that every Galois-invariant line bundle on $(C_2)_{\Qbar}$ descends
to $\QQ$~\cite[\S5.5]{Poonen2017}. In particular, $\delta_{C_2}$ is
identically zero on $J[2](C_2)(\QQ)$.

Since $\varphi_*(\eta) \in J[2](C_2)(\QQ)$ and $\delta_{C_2}(\varphi_*(\eta)) = 0$,
the \'etale double cover $D' \to C_2$ corresponding to $\varphi_*(\eta)$
descends to $\QQ$---both the cover map and the total space $D'$.
Over $\Qbar$, $D' \cong D$ (via $\varphi$), so $D$ has a $\QQ$-model.
\end{proof}

\begin{remark}
This shows that the residual gerbe of the moduli stack $\mathscr{M}_5$ at
the point $[D]$ is \emph{trivial}: $D$ does admit a $\QQ$-model.
What $\delta(\eta) \neq 0$ obstructs is only the descent of the
\emph{cover} $D \to C$, not the descent of $D$ as a curve.
\end{remark}

\begin{remark}
The $J[2](\QQ)$ subspaces of $C$ and $C_2$ are \emph{not} equal under
$\varphi_*$: the intersection $\varphi_*(J[2](C)(\QQ)) \cap J[2](C_2)(\QQ)$
has dimension~$2$ over $\FF_2$. The specific class $\eta$ survives
(i.e., $\varphi_*(\eta) \in J[2](C_2)(\QQ)$), but some elements of
$V_{\mathrm{rat}}(C)$ do \emph{not} remain rational on $C_2$.
\end{remark}

\begin{remark}
No quadric decomposition of $x^4 + y^4 + z^4$ producing a class
outside $V_{\mathrm{rat}}$ exists over~$\QQ$ itself. However,
decompositions producing $\eta$ exist over \emph{every} imaginary
quadratic field in which~$2$ does not split
(see Remark~\ref{rem:splitting_fields}), reflecting the fact that
each such field kills the Brauer class $\delta(\eta)$.
\end{remark}

%% ========================================================================
\section{A generic quartic with phantom $2$-torsion}
\label{sec:phantom}
%% ========================================================================

The Fermat quartic has a large automorphism group ($|\Aut(C_{\Qbar})| = 96$),
and one may ask whether the phenomenon of phantom $2$-torsion---i.e., a class
$\eta \in J[2](\QQ) \setminus V_{\mathrm{rat}}$ with $\delta(\eta) \neq 0$---is
special to curves with extra symmetry. We now exhibit a smooth plane quartic
$C'$ with $\Aut(C'_{\Qbar}) = 1$ (no geometric automorphisms) that possesses
phantom $2$-torsion with a \emph{different} Brauer class from the Fermat quartic.

%% ------------------------------------------------------------------------
\subsection{The construction}
\label{sec:phantom_construction}
%% ------------------------------------------------------------------------

The key observation is that the form $f = A^2 + 3B^2 + 3D^2$, for quadratic
forms $A, B, D \in \QQ[x,y,z]$, automatically carries a nontrivial Brauer
obstruction when the curve $C'\colon f = 0$ is smooth.

Over $K = \QQ(\sqrt{-3})$ with $w = \sqrt{-3}$, set
\[
  Q_1 = A + wB, \quad Q_3 = A - wB, \quad Q_2 = wD.
\]
Then $Q_1 Q_3 = A^2 + 3B^2$ and $Q_2^2 = -3D^2$, so
$f = Q_1 Q_3 - Q_2^2$ gives a quadric decomposition over~$K$.

The descent cocycle is computed as follows. Set $h = q_1/q_2$ in $K(C')$.
Since $\sigma(Q_2) = -wD = -Q_2$, we have
$\sigma(h) = \sigma(q_1)/\sigma(q_2) = q_3/(-q_2)$, and
\begin{equation}\label{eq:lambda_phantom}
  \lambda = h \cdot \sigma(h) = \frac{q_1}{q_2} \cdot \frac{-q_3}{q_2}
    = -\frac{q_1 q_3}{q_2^2} = -1,
\end{equation}
where the last step uses $q_1 q_3 = q_2^2$ on~$C'$.

Since $\lambda = -1 < 0$ and the norm form
$\Nm_{K/\QQ}(a + bw) = a^2 + 3b^2 \geq 0$, the cocycle is \emph{not} a norm.
The resulting Brauer class is
\[
  \delta(\eta) = (-1,\, -3)_\QQ,
\]
which has local invariants $\operatorname{inv}_v = 1/2$ at $v = \infty$ and
$v = 3$, and $\operatorname{inv}_v = 0$ at all other places.

\begin{remark}\label{rem:lambda_warning}
The construction $f = A^2 + 3B^2 - Q_2^2$ with $Q_2 \in \QQ[x,y,z]$
(i.e., $Q_2$ defined over~$\QQ$) always yields $\lambda = +1$: since
$\sigma(Q_2) = Q_2$, one has
$\lambda = q_1 q_3 / q_2^2 = 1$ on~$C'$. The nontrivial cocycle requires
$Q_2 = wD$ with $D$ rational, so that $\sigma(Q_2) = -Q_2$.
\end{remark}

%% ------------------------------------------------------------------------
\subsection{An explicit example}
\label{sec:phantom_example}
%% ------------------------------------------------------------------------

Taking
\[
  A = x^2 - xy - xz + y^2 - yz + z^2, \quad B = xy, \quad D = x^2 - z^2,
\]
the quartic
\begin{equation}\label{eq:phantom_quartic}
  f = A^2 + 3B^2 + 3D^2
    = 4x^4 - 2x^3y - 2x^3z + 6x^2y^2 - 3x^2z^2 - 2xy^3 - 2xz^3
      + y^4 - 2y^3z + 3y^2z^2 - 2yz^3 + 4z^4
\end{equation}
defines a smooth plane quartic $C'$ of genus~$3$ with the following properties:
\begin{enumerate}[label=(\roman*)]
  \item \emph{Positive definite}: the minimum of $f$ on the unit sphere is
    approximately $0.117 > 0$, so $C'(\RR) = \varnothing$.
  \item \emph{Trivial automorphism group}: $|\Aut(C'_{\FF_p})| = 1$ for all
    $14$ good primes $p \in \{7, 11, 13, 17, 19, 23, 29, 31, 37, 41, 43, 47,
    53, 59\}$. Since $C'$ is a non-hyperelliptic genus-$3$ curve, every
    geometric automorphism acts on the canonical space
    $H^0(C', \Omega^1) \cong \Qbar^3$ with eigenvalues that are $d$-th roots
    of unity (for $d$ the order), hence is defined over a subfield of
    $\QQ(\zeta_d)$ and is $\FF_p$-rational whenever $p \equiv 1 \pmod{d}$.
    By the Wiman bound $d \leq 4g + 2 = 14$, and for each
    $d \in \{2, \ldots, 14\}$ our list contains a prime $p \equiv 1 \pmod{d}$
    (e.g., $7$ for $d \mid 6$;  $13$ for $d \mid 12$;  $17$ for $d \mid 8$;
    $11$ for $d \mid 10$;  $29$ for $d \mid 14$;  $23$ for $d = 11$;
    $53$ for $d = 13$).
    Therefore $\Aut(C'_{\Qbar}) = 1$.
  \item \emph{Nontrivial $J[2](\QQ)$}: the $2$-adic valuation
    $v_2(\#J(C'/\FF_p))$ is at least~$1$ for all good primes
    $p \leq 53$, with $v_2(\#J(C'/\FF_7)) = 1$.
    Hence $J[2](\QQ) \cong \ZZ/2\ZZ$, generated by~$\eta$.
  \item \emph{Nontrivial $\eta$}: the divisor $\Div(q_1)$ has all-even
    multiplicities, and $\dim L(\tfrac{1}{2}\Div(q_1)) = 0$ in the function
    field of $C'$ over $K$, confirming $\eta \neq 0$ in~$J$.
  \item \emph{Phantom}: $\lambda = -1$ by~\eqref{eq:lambda_phantom},
    so $\delta(\eta) \neq 0$.
  \item \emph{No rational bitangent lines}: an exhaustive search with
    integer coefficients bounded by~$5$ finds none,
    so $V_{\mathrm{rat}} = 0$.
  \item \emph{Bad primes}: $p = 2$ and $p = 3$ only.
\end{enumerate}

\begin{proposition}\label{prop:phantom}
Let $C'\colon f = 0$ be the quartic~\eqref{eq:phantom_quartic}. Then
$\Aut(C'_{\Qbar}) = 1$, $C'(\RR) = \varnothing$, $V_{\mathrm{rat}} = 0$,
and $J[2](\QQ) = \langle \eta \rangle \cong \ZZ/2\ZZ$ with
$\delta(\eta) = (-1, -3)_\QQ \neq 0$.
\end{proposition}

In particular, the \'etale double cover $D' \to C'$ corresponding to $\eta$
does not descend to~$\QQ$. Moreover, $D'$ does not admit \emph{any}
$\QQ$-model, even as an abstract curve: since $\Aut(C'_{\Qbar}) = 1$,
the geometric automorphism group $\Aut(D'_{\Qbar}) = \langle \iota \rangle
\cong \ZZ/2\ZZ$ (the deck involution alone), and the obstruction to
descending $D'$ as an abstract curve lies in
$H^2(G_\QQ,\, \Aut(D'_{\Qbar})) \cong \Br(\QQ)[2]$.
The class $\delta(\eta) \neq 0$ in this group is the \emph{complete}
obstruction---unlike the Fermat quartic, where $|\Aut(D_{\Qbar})| = 192$
provided enough room to absorb the descent cocycle via a twist
(Remark~\ref{rem:inflation}).

%% ------------------------------------------------------------------------
\subsection{Comparison of Brauer classes}
\label{sec:phantom_brauer}
%% ------------------------------------------------------------------------

The Brauer class of the phantom quartic differs from that of the Fermat quartic:
\[
  \begin{array}{lcccc}
    & \text{Brauer class} & \operatorname{inv}_\infty
      & \operatorname{inv}_2 & \operatorname{inv}_3 \\[4pt]
    \text{Fermat:} & (-\tfrac{2}{3},\, -3)_\QQ = (-1,\, -1)_\QQ
      & \tfrac{1}{2} & \tfrac{1}{2} & 0 \\[2pt]
    \text{Phantom:} & (-1,\, -3)_\QQ
      & \tfrac{1}{2} & 0 & \tfrac{1}{2}
  \end{array}
\]
Both classes are nontrivial elements of $\Br(\QQ)[2]$, but they are ramified
at different finite primes ($2$ vs.~$3$).

Both classes are split by $\QQ(i)$, but for different reasons:
\begin{itemize}
  \item \emph{Fermat}: the class $(-1, -1)_\QQ$ has
    $\operatorname{inv}_2 = \tfrac{1}{2}$. Since $2$ ramifies in $\QQ(i)$,
    the local extension $\QQ_2(i)/\QQ_2$ has degree~$2$ and kills
    $\Br(\QQ_2)[2]$.
  \item \emph{Phantom}: the class $(-1, -3)_\QQ$ has
    $\operatorname{inv}_3 = \tfrac{1}{2}$. Since $-1$ is a quadratic
    non-residue mod~$3$, the prime $3$ is inert in $\QQ(i)$,
    so $\QQ_3(i)/\QQ_3$ is the unramified quadratic extension and kills
    $\Br(\QQ_3)[2]$.
\end{itemize}
In both cases, $\QQ(i)$ is imaginary (killing $\operatorname{inv}_\infty$)
and the relevant finite prime does not split (killing the finite invariant).
In particular, $\eta$ is representable over $\QQ(i)$ for both curves.

\begin{remark}\label{rem:no_Qi_decomp}
For the Fermat quartic, a $\QQ(i)$-rational quadric decomposition producing
$\eta$ exists explicitly (equation~\eqref{eq:decomp_Qi}). For the phantom
quartic~\eqref{eq:phantom_quartic}, however, a $\QQ(i)$-rational quadric
decomposition $f = (P + iR)(P - iR) - S^2$ with $P, R, S \in \QQ[x,y,z]_2$
would require $f + S^2 = P^2 + R^2$, a representation as a sum of two
rational squares. A computational search over $\FF_5$ (exhaustive, $5^6$
candidates for~$S$) and $\FF_{13}$ (structured search) finds no such
decomposition.

The class $\eta$ is nevertheless representable over $\QQ(i)$: the vanishing
of the Brauer obstruction guarantees the existence of a $\QQ(i)$-rational
line bundle, even without an explicit quadric decomposition.
\end{remark}

\begin{remark}\label{rem:splitting_fields_phantom}
The Brauer class $(-1, -3)_\QQ$ is split by a quadratic extension
$\QQ(\sqrt{d})$ if and only if $d$ kills both ramified places:
\begin{itemize}
  \item $\operatorname{inv}_\infty = \tfrac{1}{2}$: the extension must be
    imaginary ($d < 0$).
  \item $\operatorname{inv}_3 = \tfrac{1}{2}$: the prime $3$ must not split
    in $\QQ(\sqrt{d})$, i.e., $d \not\equiv 1 \pmod{3}$.
\end{itemize}
Examples: $\QQ(i)$ ($d = -1 \equiv 2 \pmod{3}$),
$\QQ(\sqrt{-2})$ ($d = -2 \equiv 1 \pmod{3}$---does \emph{not} work),
$\QQ(\sqrt{-3})$ ($d = -3$, $3$ ramifies).
Thus $\QQ(\sqrt{-3})$ splits the class (as expected, since the decomposition
is defined over $\QQ(\sqrt{-3})$), and $\QQ(i)$ splits it (since $-1$ is a
square), but $\QQ(\sqrt{-2})$ does not ($-2 \equiv 1 \pmod{3}$ means $3$
splits in $\QQ(\sqrt{-2})$).
\end{remark}

%% ========================================================================
%% REFERENCES
%% ========================================================================

\begin{thebibliography}{99}

\bibitem{Bruin2008}
N.~Bruin,
\emph{The arithmetic of Prym varieties in genus 3},
Compos.\ Math.\ \textbf{144} (2008), no.~2, 317--338.

\bibitem{GilleSzamuely2006}
P.~Gille and T.~Szamuely,
\emph{Central Simple Algebras and Galois Cohomology},
Cambridge Studies in Advanced Mathematics, vol.~101,
Cambridge University Press, 2006.

\bibitem{Grothendieck1968}
A.~Grothendieck,
\emph{Le groupe de Brauer I, II, III},
in \emph{Dix expos\'es sur la cohomologie des sch\'emas},
North-Holland, Amsterdam, 1968, pp.~46--188.

\bibitem{LMFDB}
The LMFDB Collaboration,
\emph{The {L}-functions and modular forms database},
\url{https://www.lmfdb.org}, 2024.

\bibitem{Koblitz1993}
N.~Koblitz,
\emph{Introduction to Elliptic Curves and Modular Forms},
2nd ed., Graduate Texts in Mathematics, vol.~97,
Springer-Verlag, New York, 1993.

\bibitem{Magma}
W.~Bosma, J.~Cannon, and C.~Playoust,
\emph{The {M}agma algebra system. {I}. {T}he user language},
J.\ Symbolic Comput.\ \textbf{24} (1997), 235--265.

\bibitem{Neukirch2008}
J.~Neukirch, A.~Schmidt, and K.~Wingberg,
\emph{Cohomology of Number Fields},
2nd ed., Grundlehren der mathematischen Wissenschaften, vol.~323,
Springer-Verlag, Berlin, 2008.

\bibitem{Poonen2017}
B.~Poonen,
\emph{Rational Points on Varieties},
Graduate Studies in Mathematics, vol.~186,
American Mathematical Society, Providence, RI, 2017.

\bibitem{Serre1979}
J.-P.~Serre,
\emph{Local Fields},
Graduate Texts in Mathematics, vol.~67,
Springer-Verlag, New York, 1979.

\bibitem{Skorobogatov2001}
A.~Skorobogatov,
\emph{Torsors and Rational Points},
Cambridge Tracts in Mathematics, vol.~144,
Cambridge University Press, 2001.

\bibitem{Zarhin2000}
Yu.~G.~Zarhin,
\emph{Hyperelliptic Jacobians without complex multiplication},
Math.\ Res.\ Lett.\ \textbf{7} (2000), no.~1, 123--132.

\end{thebibliography}

\end{document}
